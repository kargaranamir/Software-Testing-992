\documentclass[twoside]{article}
\usepackage[a4paper,total={6.5in, 10in},right=2cm,left=2cm,top=1.5cm,bottom=1.5cm,headheight=17pt, includehead,includefoot,]{geometry}
\usepackage[bottom,hang]{footmisc}
\usepackage{tasks}
\usepackage{amsmath}
\usepackage{amsfonts}
\usepackage{graphicx}
\usepackage[plain]{algorithm}
\usepackage[inline]{enumitem}
\usepackage{algpseudocode}
\usepackage{ragged2e}
\usepackage[usenames,dvipsnames]{color}
\usepackage{xcolor}
\usepackage[most]{tcolorbox}
\usepackage{hyperref}
\usepackage{dirtree}
\usepackage{textpos}
\usepackage{fancyhdr}
\usepackage{lastpage}
\usepackage[newfloat]{minted}
\usepackage{caption}
\usepackage{xepersian}

\newcommand{\Institute}{دانشکده مهندسی کامپیوتر}
\newcommand{\Title}{تمرین اول}
\newcommand{\DueDate}{۱۴۰۰/۰۱/۱۴}
\newcommand{\Class}{آزمون نرم‌افزار}
\newcommand{\Semester}{۲-۹۹}
\newcommand{\Instructor}{دکتر حسن میریان}
\newcommand{\ClassCode}{40828}


\fancyhf{}
\fancyhead[RE,LO]{\textbf{\Title}}
\fancyhead[LE,RO]{\rule[-1.4ex]{0pt}{2ex} \textbf{\Class\ {\small (\ClassCode)}}}
\fancyfoot[L]{صفحه \textbf{\thepage} از \textbf{\pageref*{LastPage}}}
\renewcommand{\footrulewidth}{0pt}

\fancypagestyle{plain}{%
	\fancyhf{} % clear all header and footer fields
	%\rule[1ex]{0pt}{2ex}
	\fancyfoot[L]{صفحه \textbf{\thepage} از \textbf{\pageref*{LastPage}}}
	\renewcommand{\headrulewidth}{0pt}
	\renewcommand{\footrulewidth}{0pt}
}

\pagestyle{fancy}

\SetupFloatingEnvironment{listing}{name={کد}}
\setlength{\footnotemargin}{2mm}
\setlist{itemsep=1pt}

\graphicspath{{./images/}}
\DeclareGraphicsExtensions{.pdf,.jpeg,.jpg,.png}

\delimitershortfall=-1pt
\hypersetup{
	colorlinks=true,
	linkcolor=blue,
	filecolor=,
	urlcolor=blue,
}

\settextfont[Scale=1.1]{HM XKayhan}
\setlatintextfont[Scale=1.1]{Linux Libertine}
\DefaultMathsDigits
\setlength{\skip\footins}{1.2pc plus 5pt}
\setlength{\parskip}{.6em}

%\setlist[enumerate,1]{label={(\arabic*)}}

\definecolor{light-gray}{gray}{0.95}
\definecolor{lighter-blue}{HTML}{EDF1F4}
\definecolor{light-blue}{HTML}{12558E}
\definecolor{light-orange}{HTML}{F4F0ED}
%EDE4DD

\linespread{1.5}

\setlength\parindent{0pt}

\newenvironment{answer}{}{\medskip}
\newenvironment{question}[1]{\textbf{#1.} }{}
\tcolorboxenvironment{question}{enhanced,size=normal,colback=lighter-blue,boxrule=0pt,sharp corners}

\newenvironment{qitem}[1]{%
	\begin{enumerate}\item[(\lr{#1})]}{\end{enumerate}}
\tcolorboxenvironment{qitem}{enhanced,size=normal,colback=light-orange,boxrule=0pt,sharp corners}

\newtcbox{\code}{on line,fontupper=\ttfamily,size=fbox,colback=light-gray,boxrule=0pt,sharp corners}

\newtcolorbox{titlebox}{enhanced,size=normal,colback=light-gray,colframe=black,sharp corners,shadow={-1mm}{-1mm}{0mm}{black},boxrule=0.6pt,boxsep=2mm}
\newtcolorbox{hints}[1][]{colback=blue!10!white,colframe=blue!50!black,title=#1,fonttitle=\bfseries\large}


\definecolor{pblue}{rgb}{0.13,0.13,1}
\definecolor{pgreen}{rgb}{0,0.5,0}
\definecolor{pred}{rgb}{0.9,0,0}
\definecolor{pgrey}{rgb}{0.46,0.45,0.48}
\definecolor{customblue}{RGB}{235,241,245}

\newmintedfile[Javacode]{java}{
	fontfamily=tt,
	fontsize=\footnotesize,
	bgcolor=customblue,
	formatcom=\setLTR,
	linenos=true,
	numberblanklines=true,
	numbersep=5pt,
	escapeinside=||,
	gobble=0,
	%frame=leftline,
	frame=single,
	framerule=0.4pt,
	framesep=3mm,
	funcnamehighlighting=true,
	tabsize=3,
	obeytabs=false,
	mathescape=false,
	samepage=false, %with this setting you can force the list to appear on the same page
	showspaces=false,
	showtabs =false,
	texcl=false,
	breaklines=true,
	baselinestretch=1.1
}

\begin{document}
	\thispagestyle{plain}
	\begin{textblock*}{\textwidth}(0pt,-1.5cm)
		\centering \small
		بسمه تعالی
	\end{textblock*}
	\begin{minipage}[c]{0.7\textwidth}
		\flushright
		{\Large\bfseries{\Class} (\ClassCode)\ {\scriptsize (نیم‌سال \Semester)}}\\
		{\small \Instructor}
	\end{minipage}
	\hfill
	\begin{minipage}[c]{0.17\textwidth}
		\centering
		\includegraphics[width=.6\textwidth]{sharif-logo}\\[-5pt]
		{\scriptsize \Institute}
	\end{minipage}
	\vspace{2ex}
	\begin{titlebox}
		{\Large{\textbf{\Title}}}\hfill
		\parbox{.4\textwidth}{%
			\vspace{-1.3ex}
			\flushleft
			\small
امیرحسین کارگران خوزانی {\footnotesize (۹۹۲۰۱۱۱۹)}\\
سید سجاد میرزابابایی {\footnotesize (۹۹۲۱۰۱۴۲)}\\
رامتین باقری {\footnotesize (۹۹۳۰۱۹۳۸)}
		}
	\end{titlebox}
% ~~~~~~~~~~~~~~~~~~~~~~~~~~~~~~~~~~~~~~~~~~~
\subsection*{بخش نظری}
\begin{question}{1}
در این بخش باید یک برنامه تبدیل اعداد رومی به اعداد دهدهی را مورد بررسی قرار دهید. این برنامه به زبان برنامه‌نویسی جاوا نوشته شده است.
\end{question}
\begin{qitem}{a}
خطا یا خطاهای برنامه در کدام قسمت هستند؟ اصلاح‌شده آنها را بنویسید.
\end{qitem}
\begin{answer}
خروجی مورد انتظار این تابع برای ورودی \code{II} مقدار \code{2} خواهد بود؛ اما پس از اجرای تابع با ورودی مذکور، مقدار \code{0} نتیجه می‌شود. این مشکل از آنجا سرچشمه می‌گیرد که افزودن عدد جدید به اعداد قبلی \textbf{تنها} در صورتی انجام می‌شود که شرط \code{next > currentNumber} برقرار باشد، با این حال نیاز است تا برابری این دو متغیر نیز برای افزودن عدد لحاظ شود.
چرا که در غیر این صورت در اعدادی مانند \code{II} که یک کاراکتر پشت‌سرهم تکرار شده است خطا ظاهر می‌شود.
با تغییر در خط ۲۳ و اضافه کردن حالت برابری دو متغیر، مشکل حل خواهد شد. نسخه اصلاح‌شده برنامه در \autoref{code:convert} موجود است.

نکته‌ای که باید به آن توجه داشت این است که تمام رشته‌های شامل ترکیب حروف استفاده شده در اعداد رومی الزاما صحیح نیستند؛ برای مثال، عدد \code{8} تنها به فرم \code{VIII} مورد قبول است و اگر به شکل \code{IIX} نوشته شود صحیح نیست.
از آنجا که دقیقا در توصیف برنامه گفته نشده است که در این شرایط چه باید کرد می‌توان دو فرض داشت:
\begin{enumerate}
	\item
حتما ورودی به فرم اعداد رومی است و ورودی‌ غیر مجازی داده نمی‌شود. در این صورت تنها خطای برنامه، همان خطای نامساوی بیان شده است.
	\item
اگر ورودی به فرم اعداد رومی نباشد آن‌گاه باید برنامه خطای مناسب را در خروجی نشان دهد. در این صورت برنامه یک خطای دیگر نیز دارد چرا که این برنامه‌ برای تمام رشته‌های بدفرم نیز جوابی در سیستم اعداد دهدهی تولید می‌‌کند.
\end{enumerate}
به منظور جلوگیری از این اتفاق، می‌توان با استفاده از عبارت منظم زیر چک کرد که رشته ورودی معتبر است یا خیر. و تنها در صورتی که معتبر بود برنامه به شیوه‌ای که اصلاح شده است ادامه یابد و در غیر این صورت در خروجی، خطای مناسب تولید شود.

تنها در صورتی متغیر \code{valid} برابر \code{1} خواهد شد که رشته مورد نظر با یکی از فرم‌های صحیح عبارت منظم زیر تطبیق یابد:
\code{\lr{\texttt{boolean valid = word.matches("\^M\{0,4\}(CM|CD|D?C\{0,3\})(XC|XL|L?X\{0,3\})(IX|IV|V?I\}0,3\})\$");}}}
\end{answer}
\begin{listing}[h]
\caption{نسخه اصلاح‌شده برنامه تبدیل اعداد رومی به دهدهی}
\Javacode{resources/prg.java}
\label{code:convert}
\end{listing}
% ====================================================
\begin{qitem}{b}
در صورت امکان مورد آزمونی ارائه دهید که خطا را اجرا نکند.
\end{qitem}
\begin{answer}
از آنجایی محل خطا در خط شماره ۲۳ قرار داد، باید مورد آزمون را به نحوی طراحی کنیم که این خط از کد اجرا نشود. از این رو مورد آزمون هدف، دارای ورودی \code{""} (رشته متنی با طول صفر) و خروجی قابل انتظار \code{0} است. چرا که طول این رشته برابر صفر می‌باشد و شرط \code{s.length() < i} در حلقه \code{for} که مقدار اولیه \code{i} برابر \code{0} است ارضا نمی‌شود و این حلقه اجرا نمی‌گردد.
در بقیه حالات خط شماره ۲۳ که محل خطاست اجرا می‌گردد.
\end{answer}
% ====================================================
\begin{qitem}{c}
در صورت امکان آزمونی بنویسید که خطا را اجرا کند اما نتیجه حالت میانی و پایانی مشخص کننده حالت اشتباه نباشد.
\end{qitem}
\begin{answer}
همانطور که بالاتر گفته شد، با اجرا شدن خط شماره ۲۳، خطا نیز اجرا خواهد شد. مورد آزمون هدف، ورودی \code{IV} و خروجی قابل انتظار \code{4} را خواهد داشت. در این صورت خطا اجرا می‌شود ولی تاثیری بر فرآیند محاسبه نخواهد گذاشت.
\end{answer}
% ====================================================
\begin{qitem}{d}
در صورت امکان آزمونی بنویسید که برنامه در حالت میانی اشتباه است اما در پایان نتیجه شکست نمی‌شود. با مثالی نتیجه مورد انتظار و نتیجه اجرا را نشان دهید.
\end{qitem}
\begin{answer}
برنامه زمانی دارای حالت میانی اشتباه می‌شود که هر دو مورد زیر اتفاق بیفتد:
\begin{enumerate}
	\item خط شماره ۲۳ اجرا شود
	\item ورودی دارای حداقل دو کاراکتر یکسان متوالی باشد.
\end{enumerate}

همان‌گونه که بحث شد تنها درصورتی که حداقل دو کاراکتر یکسان متوالی وجود داشته باشد، به جای شرط \code{if}، شرط \code{else} اجرا می‌شود. از آنجا که این مورد باعث می‌شود که تنها به غلط از مقدار صحیح کم شود و جای دیگری در برنامه به غلط چیزی اضافه نمی‌شود، پس این حالت نمی‌تواند رخ دهد و نمی‌توان طبیعتا برای آن مورد آزمونی نیز که آن را ارضا کند نوشت.

اما اگر ورودی‌های بدفرم رشته اعداد رومی که شامل کاراکترهای مجاز آن هستند را این تابع به عنوان ورودی بپذیرد ان‌گاه اگر تفسیر رشته‌ای به فرم \code{IIX} (شامل دو کاراکتر تکراری متوالی قبل از یک کاراکتر پرارزش‌تر) را برابر \code{8 = 2 - 10} در نظر بگیریم (که این کار اشتباه است، و صرفا اینجا مشاهده را گزارش کرده‌ایم). علی رغم اشتباه بودن حالت میانی در حین محاسبه، نتیجه نهایی برابر \code{8} خواهد بود.
\end{answer}
% ====================================================
\begin{qitem}{e}
با استفاده از تحلیل آر.آی.پی شرایطی را تعیین کنید که مورد آزمون تشخیص دهنده خطای این برنامه باید داشته باشد.
\end{qitem}
\begin{answer}
نتیجه تحلیل \autoref{code:convert} با استفاده از مدل \lr{RIP}، به شکل زیر است:

\begin{table}[h]
	\centering
	\begin{tabular}{c|c}
		دسترسی‌پذیری & $|S| > 0$\\\hline
		آلودگی & $\exists i, 0 \leq i < i + 1 < |S| \wedge S[i]=S[i+1]$\\\hline
		انتشار &
		$\exists i, 0 \leq i < i + 1 < |S| \wedge S[i]=S[i+1]$
	\end{tabular}
\end{table}
بنابراین، مشخصات آزمون تشخیص دهنده خطای برنامه عبارت است از:
\begin{align*}
	|S| > 0\ \wedge \left(\exists i, j\ 0 \leq i < j < |S| \wedge j - i = 1 \wedge S[i]=S[j]\right)
\end{align*}
\end{answer}
% ====================================================
\begin{question}{2}
برنامه داده شده، کوچکترین و بزرگترین عنصر آرایه‌ای از اعداد صحیح را پیدا می‌کند. پیاده‌سازی این برنامه اشتباه است. با استفاده از روش افراز فضای ورودی، موارد آزمونی را طراحی کنید که خطای برنامه را مشخص کند. رویکرد شما باید حداقل دو خصوصیت مبتنی بر عملکرد و یک خصوصیت مبتنی بر واسط داشته باشد.
\end{question}
\begin{listing}[h]
	\caption{برنامه یافتن بزرگترین و کوچکترین اعداد آرایه}
	\Javacode{resources/q2.java}
	\label{code:q2}
\end{listing}
\begin{answer}
خروجی مورد انتظار این برنامه بیشترین و کمترین مقدار آرایه‌ی \code{nums} است.
در ادامه با استفاده از روش افراز فضای ورودی خطای این برنامه مشخص می‌شود. و در می‌یابیم که اگر ورودی را به نحوی ارائه دهیم که جایگاه بزرگترین داده قبل از جایگاه بقیه داده‌ها در لیست باشد، آنگاه مقدار \code{largest} هیچ‌گاه برابر بزرگترین داده نخواهد بود.
برای مثال لیستی با اعضای \code{\lr{[4, 3, 2, 1]}}
را در نظر بگیرید، آنگاه خروجی مورد انتظار برای \code{smallest} \code{1} و برای \code{largest}، مقدار \code{4} است. پس از اجرا در می‌یابیم که خروجی
 کد به ازای این لیست برای متغیر \code{smallest} مقدار \code{1} و برای متغیر \code{largest} مقدار \code{-‌2147483648} خواهد بود. 
 نسخه اصلاح شده برنامه در \autoref{code:q2} آمده است. 

\subsubsection*{خصوصیات مبتنی بر واسط}
\begin{enumerate}
	\item[\lr{\textbf{A}}.]
علامت متغیر \code{smallest}
	\begin{tasks}(3)
		\task[\lr{1}. ] مثبت
		\task[\lr{2}. ] منفی
		\task[\lr{3}. ] صفر
	\end{tasks}
	\item[\lr{\textbf{B}}.]
علامت متغیر \code{largest}
	\begin{tasks}(3)
		\task[\lr{1}. ] مثبت
		\task[\lr{2}. ] منفی
		\task[\lr{3}. ] صفر
	\end{tasks}
	\item[\lr{\textbf{C}}.]
اندازه لیست \code{nums}
	\begin{tasks}(3)
		\task[\lr{1}. ] صفر
		\task[\lr{2}. ] یک 
		\task[\lr{3}. ] بیشتر از یک
	\end{tasks}
	\item[\lr{\textbf{D}}.]
آیا لیست \code{nums} به ترتیب صعودی مرتب شده است؟
	\begin{tasks}(2)
		\task[\lr{1}. ] بله
		\task[\lr{2}. ] خیر 
	\end{tasks}
	\item[\lr{\textbf{E}}.]
آیا لیست \code{nums} به ترتیب نزولی مرتب شده است؟
	\begin{tasks}(2)
		\task[\lr{1}. ] بله
		\task[\lr{2}. ] خیر 
	\end{tasks}
	\item[\lr{\textbf{F}}.]
آیا تمامی اعضای \code{nums} از جنس  \texttt{int} (در بازه مجاز) هستند؟
	\begin{tasks}(2)
		\task[\lr{1}. ] بله
		\task[\lr{2}. ] خیر
	\end{tasks}
	\item[\lr{\textbf{G}}.]
آیا \code{n} \texttt{null} است یا خیر؟
	\begin{tasks}(2)
		\task[\lr{1}. ] بله
		\task[\lr{2}. ] خیر
	\end{tasks}
	\item[\lr{\textbf{H}}.]
آیا \code{n} نوع \texttt{int} است؟
	\begin{tasks}(2)
		\task[\lr{1}. ] بله
		\task[\lr{2}. ] خیر
	\end{tasks}
	\item[\lr{\textbf{I}}.]
آیا \code{n} در بازه ی مجاز \texttt{int} است یا خیر؟
	\begin{tasks}(2)
		\task[\lr{1}. ] بله
		\task[\lr{2}. ] خیر
	\end{tasks}
\end{enumerate}
\subsubsection*{خصوصیات مبتنی بر عملکرد}
\begin{enumerate}
	\item[\lr{\textbf{J}}.]
آیا بزرگترین عنصر قبل از تمام عنصرهای دیگر در لیست ظاهر شده است؟
	\begin{tasks}(2)
		\task[\lr{1}. ] بله
		\task[\lr{2}. ] خیر
	\end{tasks}
	\item[\lr{\textbf{K}}.]
آیا کوچکترین عنصر قبل از تمام عنصرهای دیگر در لیست ظاهر شده است؟
	\begin{tasks}(2)
		\task[\lr{1}. ] بله
		\task[\lr{2}. ] خیر
	\end{tasks}
\end{enumerate}
حال با توجه به خصوصیات مطرح شده و افراز بلاک‌های انجام شده، به روش \lr{BCC}%
\LTRfootnote{Bace Choice Coverage}
اقدام به ساخت نیاز و موارد آزمون می‌کنیم که لیست این نیازها در \autoref{tbl:bcc-test-cases} آورده شده است. همچنین موارد آزمون نیز در جدول \autoref{tbl:bcc-test-cases-2} نمایش داده شده است. در این روش کافی است یک پایه در نظر بگیریم و هر بار یکی از خصوصیت‌ها تغییر پیدا کند.

	\begin{table}[h]
		\centering
		\caption{موارد آزمون ساخته شده به روش \lr{BCC}}
		\label{tbl:bcc-test-cases}
		\begin{latin}
		\begin{tabular}{l|c|c|c|c|c|c|c|c|c|c|c|c|c|c|c|}
\cline{2-16}
                        & \multicolumn{1}{l|}{\textbf{BASE}} & \multicolumn{1}{l|}{\textbf{TC1}} & \multicolumn{1}{l|}{\textbf{TC2}} & \multicolumn{1}{l|}{\textbf{TC3}} & \multicolumn{1}{l|}{\textbf{TC4}} & \multicolumn{1}{l|}{\textbf{TC5}} & \multicolumn{1}{l|}{\textbf{TC6}} & \multicolumn{1}{l|}{\textbf{TC7}} & \multicolumn{1}{l|}{\textbf{TC8}} & \multicolumn{1}{l|}{\textbf{TC9}} & \multicolumn{1}{l|}{\textbf{TC10}} & \multicolumn{1}{l|}{\textbf{TC11}} & \multicolumn{1}{l|}{\textbf{TC12}} & \multicolumn{1}{l|}{\textbf{TC13}} & \multicolumn{1}{l|}{\textbf{TC14}} \\ \hline
\multicolumn{1}{|l|}{A} & A1                                 & A2                                & A3                                & A1                                & A1                                & A1                                & A1                                & A1                                & A1                                & A1                                & A1                                 & A1                                 & A1                                 & A1                                 & A1                                 \\ \hline
\multicolumn{1}{|l|}{B} & B1                                 & B1                                & B1                                & B2                                & B3                                & B1                                & B1                                & B1                                & B1                                & B1                                & B1                                 & B1                                 & B1                                 & B1                                 & B1                                 \\ \hline
\multicolumn{1}{|l|}{C} & C3                                 & C3                                & C3                                & C3                                & C3                                & C1                                & C2                                & C3                                & C3                                & C3                                & C3                                 & C3                                 & C3                                 & C3                                 & C3                                 \\ \hline
\multicolumn{1}{|l|}{D} & D2                                 & D2                                & D2                                & D2                                & D2                                & D2                                & D2                                & D1                                & D2                                & D2                                & D2                                 & D2                                 & D2                                 & D2                                 & D2                                 \\ \hline
\multicolumn{1}{|l|}{E} & E2                                 & E2                                & E2                                & E2                                & E2                                & E2                                & E2                                & E2                                & E1                                & E2                                & E2                                 & E2                                 & E2                                 & E2                                 & E2                                 \\ \hline
\multicolumn{1}{|l|}{F} & F1                                 & F1                                & F1                                & F1                                & F1                                & F1                                & F1                                & F1                                & F1                                & F2                                & F1                                 & F1                                 & F1                                 & F1                                 & F1                                 \\ \hline
\multicolumn{1}{|l|}{G} & G2                                 & G2                                & G2                                & G2                                & G2                                & G2                                & G2                                & G2                                & G2                                & G2                                & G1                                 & G2                                 & G2                                 & G2                                 & G2                                 \\ \hline
\multicolumn{1}{|l|}{H} & H1                                 & H1                                & H1                                & H1                                & H1                                & H1                                & H1                                & H1                                & H1                                & H1                                & H1                                 & H2                                 & H1                                 & H1                                 & H1                                 \\ \hline
\multicolumn{1}{|l|}{I} & I1                                 & I1                                & I1                                & I1                                & I1                                & I1                                & I1                                & I1                                & I1                                & I1                                & I1                                 & I1                                 & I2                                 & I1                                 & I1                                 \\ \hline
\multicolumn{1}{|l|}{J} & J2                                 & J2                                & J2                                & J2                                & J2                                & J2                                & J2                                & J2                                & J2                                & J2                                & J2                                 & J2                                 & J2                                 & J1                                 & J2                                 \\ \hline
\multicolumn{1}{|l|}{K} & K2                                 & K2                                & K2                                & K2                                & K2                                & K2                                & K2                                & K2                                & K2                                & K2                                & K2                                 & K2                                 & K2                                 & K2                                 & K1                                 \\ \hline
	\end{tabular}
	\end{latin}
	\end{table}
	\begin{table}[h]
		\centering
		\caption{موارد آزمون ساخته شده به روش \lr{BCC}}
		\label{tbl:bcc-test-cases-2}
		\begin{latin}
\begin{tabular}{c|c|c|c|}
\cline{2-4}
                                    & \textbf{Input}              & \textbf{Expected Smallest}   & \textbf{Expected Largest}  \\ \hline
\multicolumn{1}{|c|}{\textbf{BASE}} & {[}3,6,2,1,4{]}             & 1                            & 6                          \\ \hline
\multicolumn{1}{|c|}{\textbf{TC1}}  & {[}3,6,2,-1,4{]}            & -1                           & 6                          \\ \hline
\multicolumn{1}{|c|}{\textbf{TC2}}  & {[}3,6,2,0,4{]}             & 0                            & 6                          \\ \hline
\multicolumn{1}{|c|}{\textbf{TC3}}  & {[}-3,-1,-7,-2{]}           & -7                           & -1                         \\ \hline
\multicolumn{1}{|c|}{\textbf{TC4}}  & {[}-3,0,-7,-2{]}            & -7                           & -1                         \\ \hline
\multicolumn{1}{|c|}{\textbf{TC5}}  & \multicolumn{3}{c|}{invalid, against constraints F ...}                                 \\ \hline
\multicolumn{1}{|c|}{\textbf{TC6}}  & \multicolumn{3}{c|}{invalid, against constraints J and K}                               \\ \hline
\multicolumn{1}{|c|}{\textbf{TC7}}  & \multicolumn{3}{c|}{invalid, against constraint K}                                      \\ \hline
\multicolumn{1}{|c|}{\textbf{TC8}}  & \multicolumn{3}{c|}{invalid, against constraint J}                                      \\ \hline
\multicolumn{1}{|c|}{\textbf{TC9}}  & {[}3,MaxInt + 1, 4,5,1,2{]} & \multicolumn{2}{c|}{program should notify user and exit.} \\ \hline
\multicolumn{1}{|c|}{\textbf{TC10}} & null                        & \multicolumn{2}{c|}{program should notify user and exit.} \\ \hline
\multicolumn{1}{|c|}{\textbf{TC11}} & {[}2.1,6.2,2.5,1.0,3.4{]}   & \multicolumn{2}{c|}{program should notify user and exit.} \\ \hline
\multicolumn{1}{|c|}{\textbf{TC12}} & {[}3,MaxInt + 1, 4,5,1,2{]} & \multicolumn{2}{c|}{program should notify user and exit.} \\ \hline
\multicolumn{1}{|c|}{\textbf{TC13}} & {[}8,3,6,2,1,4{]}           & 1                            & 8                          \\ \hline
\multicolumn{1}{|c|}{\textbf{TC14}} & {[}1,3,6,2,4{]}             & 1                            & 6                          \\ \hline
	\end{tabular}
	\end{latin}
	\end{table}
مورد آزمون شماره ی ۱۳ به اشتباه ۶ را به عنوان largest اعلام می کند که خطای برنامه را نشان می دهد و همچنین سایر موارد آزمون ۹ ، ۱۰ ، ۱۱ و ۱۲ نیز از برنامه توقع خروجی مناسب می رود تا خطای ورودی را به اطلاع کاربر برساند اما برنامه به درسی خروجی نمی دهد.
\end{answer}
\subsection*{بخش عملی}
\begin{question}{3}
هدف این بخش آن است که یک برنامه پشته را با استفاده از رویه کاری ایجاد آزمون رانه گشترش داده و مورد آزمون قرار دهید. شما باید برای این برنامه که زبان برنامه‌نویسی جاوا نوشته شده است، به شیوه ایجاد آزمون رانه قابلیت جست‌و‌جو در یک پشته را فراهم کنید. به بیان دقیق‌تر، باید یک تابع با واسط زیر را پیاده‌سازی کنید.

\end{question}
%\begin{listing}[h]
%	\caption{واسط تابع جستجو در پشته}
%	\Javacode{resources/q3.java}
%	\label{code:q3}
%\end{listing}
\begin{answer}
روش TDD یک رویکرد توسعه نرم افزار است که در آن یک تست در سطح یونیت قبل از نوشتن کد، نوشته می شود. به طور کلی، رویه ی زیر در راستای توسعه نرم افزار TDD تکرار می شود:
\begin{enumerate}
	\item
یک تست اضافه می شود.
	\item
همه ی تست ها را اجرا می کنیم. آیا تست جدید با شکست رو به رو می شود؟
	\item
مقداری کد می نویسیم تا تست شکست خورده را با موفقیت بگذراند.
	\item
تست ها را مجددا اجرا می کنیم.
	\item
کد را ریفکتور می کنیم و مجددا تست ها را اجرا کنیم تا اطمینان حاصل کنیم در طول ریفکتور خطایی وارد برنامه نشده است.
	\item
مراحل را مجددا از مرحله ی ۱ تکرار می کنیم.
\end{enumerate}
این روش نوشتن تست های ناموفق به چرخه ی "قرمز - سبز - ریفکتور" شناخته می شود شما ابتدا باید به خوبی فکر کنید که برنامه قرار است چه کاری را انجام دهد و بعد از نوشتن تست باید سعی کنی با دیباگ کد از تست با موفقیت عبور کنید و حق تغییر کد تست را نخواهید داشت. در ادامه ویژگی‌های معمول برای پشته نوشته شده و جدول موارد آزمون ساخته می‌شود. در نهایت تمامی تست‌ها توسط Juint مورد ارزیابی قرار می‌گیرند.

ویژگی‌های معمول برای پشته به صورت زیر مورد پرسش قرار می‌گیرند:
\begin{enumerate}
	\item[\lr{\textbf{A}}.]
آیا پشته خالی است؟
	\begin{tasks}(2)
		\task[\lr{1}. ]
بله
		\task[\lr{2}. ]
خیر
	\end{tasks}
	\item[\lr{\textbf{B}}.]
اندازه پشته چقدر است؟
	\begin{tasks}(3)
		\task[\lr{1}. ]
صفر
		\task[\lr{2}. ]
یک
		\task[\lr{3}. ]
بیشتر از یک
	\end{tasks}
	\item[\lr{\textbf{C}}.]
آیا پشته شامل عنصرهای \texttt{null} است؟
	\begin{tasks}(2)
		\task[\lr{1}. ]
بله
		\task[\lr{2}. ]
خیر
	\end{tasks}
\end{enumerate}
همچنین ویژگی‌های مربوط به عنصر که قصد جستجوی آن را داریم به زیر هستند:
\begin{enumerate}[label=.\Alph*]
	\item[\lr{\textbf{D}}.]
آیا عنصر \code{i}، \texttt{null} است؟
	\begin{tasks}(2)
		\task[\lr{1}. ]
بله
		\task[\lr{2}. ]
خیر
	\end{tasks}
\end{enumerate}
و همچنین سولاتی که پیرامون رابطه پشته و عنصری که قصد جستجوی آن در پشته را داریم قابل طرح هستند، نظیر:
\begin{enumerate}[label=.\Alph*]
	\item[\lr{\textbf{E}}.]
آیا عنصر \code{i} در پشته وجود دارد؟
	\begin{tasks}(2)
		\task[\lr{1}. ]
بله
		\task[\lr{2}. ]
خیر
	\end{tasks}
	\item[\lr{\textbf{F}}.]
آیا عنصر \code{i}، المان اول در پشته است؟
	\begin{tasks}(2)
		\task[\lr{1}. ]
بله
		\task[\lr{2}. ]
خیر
	\end{tasks}
	\item[\lr{\textbf{G}}.]
آیا عنصر \code{i}، المان آخر در پشته است؟
	\begin{tasks}(2)
		\task[\lr{1}. ]
بله
		\task[\lr{2}. ]
خیر
	\end{tasks}
\end{enumerate}
حال با توجه به خصوصیات مطرح شده و افراز بلاک‌های انجام شده، به روش \lr{EC}%
\LTRfootnote{Each Choice}
اقدام به ساخت نیاز آزمون می‌کنیم که لیست این نیازها در \autoref{tbl:ec-test-cases} آورده شده است. در این روش کافی است از هر بلاک حداقل یک مورد در آزمون استفاده شود.

	\begin{table}[h]
		\centering
		\caption{نیاز آزمون ساخته شده به روش \lr{EC}}
		\label{tbl:ec-test-cases}
		\begin{latin}
		\begin{tabular}{l|c|c|c|c|c|c|c|c|}
			\cline{2-9}
			& \textbf{TC 1} & \textbf{TC 2} & \textbf{TC 3} & \textbf{TC 4} & \textbf{TC 5} & \textbf{TC 6} & \textbf{TC 7} & \textbf{TC 8} \\ \hline
			\multicolumn{1}{|l|}{\textbf{A}} & A2         & A2         & A2         & A1         & A2         & A2         & A2         & A2         \\ \hline
			\multicolumn{1}{|l|}{\textbf{B}} & B3         & B3         & B3         & B1         & B2         & B3         & B3         & B3         \\ \hline
			\multicolumn{1}{|l|}{\textbf{C}} & C2         & C2         & C2         & C2         & C2         & C1         & C2         & C2         \\ \hline
			\multicolumn{1}{|l|}{\textbf{D}} & D2         & D2         & D2         & D2         & D2         & D2         & D1         & D2         \\ \hline
			\multicolumn{1}{|l|}{\textbf{E}} & E1         & E1         & E2         & E2         & E1         & E1         & E2         & E1         \\ \hline
			\multicolumn{1}{|l|}{\textbf{F}} & F1         & F2         & F2         & F2         & F1         & F2         & F2         & F2         \\ \hline
			\multicolumn{1}{|l|}{\textbf{G}} & G2         & G1         & G2         & G2         & G1         & G2         & G2         & G2         \\ \hline
		\end{tabular}
	\end{latin}
	\end{table}

توجه کنید که با توجه به اینکه زبان جاوا به صورت \lr{strict-type} است، امکان \lr{push} کردن عنصر \lr{null} به پشته وجود ندارد و همچنین در صورت ارسال آن به تابع \lr{search}، با خطای \lr{compile} روبرو خواهیم شد. به همین دلیل، موارد آزمون ۶ و ۷ در عمل پیاده‌سازی نمی‌شوند. \autoref{fig:test-cases} نتایج اعمال موارد آزمون را توسط چارچوب \lr{JUnit} نمایش می‌دهد.

\begin{figure}
	\caption{نتایج موارد آزمون اعمال شده}
	\includegraphics[width=\textwidth]{test-status-q3}
	\label{fig:test-cases}
\end{figure}
\end{answer}
\end{document}